\chapter{Introduction}
\begin{multicols}{2}
    [\section{Project Background}]
    In today's rapidly evolving digital landscape, cybersecurity remains a paramount concern for organizations
    across all industries. With the proliferation of sophisticated cyber threats and the increasing complexity of
    \acrshort{it} infrastructures, business are constantly seeking new and innovative ways to protect their digital
    assets and fortify their defences and safeguard sensitive data. In this pursuit, cybersecurity consultant firms
    have emerged as a critical ally for organizations, providing expert guidance and support in the development and
    implementation of robust cybersecurity strategies, playing a pivotal role in offering expertise and guidance to
    help organizations navigate the intricate realm of cybersecurity.

    One of the key strategies employed by cybersecurity consultants is the integration of third-party security
    \acrshort{api}s into their arsenal of tools and technologies. These \acrshort{api}s provide invaluable
    functionalities, ranging from vulnerability assessment and security scans to device health monitoring and
    threat intelligence analysis by \acrshort{ai}. By leveraging these \acrshort{api}s, cybersecurity consultants
    can enhance their capabilities and provide a more comprehensive and effective security solution to their
    clients, streamline their operations, provide clients with robust, proactive security measurers, and improve
    their overall service delivery.

    \section{Q-ICT Software Development Department}
    \acrshort{qict}, a small cybersecurity consultancy that the author is currently doing his graduation
    internship in, recognizes the critical importance of proactive \acrshort{api} monitoring in safeguarding its clients' digital
    assets. Their customers are small to medium-sized business with employees ranging from 1 to 100. \acrshort{qict} is
    therefore asked to monitor their clients' devices and ensuring the overall security of their systems, IT
    infrastructure, and digital assets. They typically engage in various activities, including:
    \begin{itemize}
        \item \textbf{Continuous Monitoring and Maintenance}: implementing tools and processes for continuous monitoring of
              clients' systems, devices, networks, and systems to detect and respond to security threats in
              real-time and address emerging threats and vulnerabilities.
        \item \textbf{Vulnerability Assessment}: conducting regular vulnerability assessments and penetration
              testing to identify weaknesses in clients' systems and infrastructure
        \item \textbf{Incident Response}: developing and implementing plans and protocols for responding to
              and mitigating cybersecurity incidents effectively and efficiently.
        \item \textbf{Penetration Testing}: simulating cyber attacks to identify weaknesses in the client's defences and
              assess their ability to withstand and respond to real-world cyber threats.
        \item \textbf{Secyrity Incident Investigation}: conducting thorough investigations into security incidents to
              identify the root cause and impact of the incident and develop strategies to prevent future occurrences.
    \end{itemize}
    The company consists of multiple departments in its behalf, each with their own functions and responsibilities.
    Those departments are the following:
    \begin{enumerate}
        \item Service Help Desk Department:
        \item Cybersecurity Department:
        \item Software Development Department:
        \item Financial Department:
    \end{enumerate}
    The author of this document is currently working within the software development department under the supervision of
    2 people, which are the senior software developer and the cybersecurity specialist.

    The company currently manages numerous third-party \acrshort{api}s for the above-mentioned purposes. Those
    \acrshort{api}s are the following:
    \begin{itemize}
        \item Pax8:
        \item SnelStart:
        \item SentinelOne:
        \item Bodyguard.io:
        \item N-Central:
        \item PerfectView:
        \item Computicate:
    \end{itemize}
    Currently, those \acrshort{api}s are managed manually and without a standardized implementation in their
    internal application, the \acrshort{qaas} app, which is a time-consuming and error-prone process. This has
    led to a several problems, namely:
    \begin{itemize}
        \item Inefficient and fragmented approach to \acrshort{api} management.
        \item Lack of user-friendliness, and slow and unclear navigation.
        \item Inconsistent integration of \acrshort{api}s into the application.
        \item Poses a significant challenge in error handling and debugging, as disparate error reporting mechanism
              across the APIs hinder efficient troubleshooting and resolution processes.
        \item Difficulty to maintain and update \acrshort{api}s.
        \item Lack of clear and concise documentation.
        \item Lack of a centralized \acrshort{api} management system.
        \item Inadequate security measures, as the company risks inconsistent data retrieval and analysis across its \acrshort{api}s,
              potentially leading to incomplete insights into client \acrshort{it} systems and infrastructure..
    \end{itemize}

    \section{Project Objectives}
    In the end of this project which consist of 90-99 working days, the following objectives should be achieved:
    \begin{enumerate}
        \item Develop a
    \end{enumerate}
    \section{Reading Guide}

\end{multicols}