\chapter{Introduction}
\begin{multicols}{2}
      [\section{Project Background}]
      In today's rapidly evolving digital landscape, cybersecurity remains a paramount concern for organizations
      across all industries. With the proliferation of sophisticated cyber threats and the increasing complexity of
      \acrshort{it} infrastructures, business are constantly seeking new and innovative ways to protect their
      digital assets and fortify their defences and safeguard sensitive data. In this pursuit, cybersecurity
      consultant firms have emerged as a critical ally for organizations, providing expert guidance and support in
      the development and implementation of robust cybersecurity strategies, playing a pivotal role in offering
      expertise and guidance to help organizations navigate the intricate realm of cybersecurity.

      One of the key strategies employed by cybersecurity consultants is the integration of third-party security
      \gls{API}s into their arsenal of tools and technologies. These \acrshort{api}s provide invaluable
      functionalities, ranging from vulnerability assessment and security scans to device health monitoring and
      threat intelligence analysis by \acrshort{ai}. By leveraging these \acrshort{api}s, cybersecurity consultants
      can enhance their capabilities and provide a more comprehensive and effective security solution to their
      clients, streamline their operations, provide clients with robust, proactive security measures, and improve
      their overall service delivery.

      SentinelOne emerges as one of the leading organization in the cybersecurity real, offering cutting-edge
      endpoint protection and threat intelligence solutions. It leverages advance \acrshort{ai} and \acrshort{ml}
      algorithms, providing comprehensive protection against malware, ransomware, and other cyber-threats. It is one
      of the must-have protection tools for business organizations looking to bolster their cybersecurity posture
      and safeguard their digital assets.

      As the main topic of this graduation work placement project, the author
      has been tasked to integrate SentinelOne endpoint protection capabilities into the client company internal
      application, the \acrshort{qaas} app. The main objective of this project is to promote transparency more
      into \acrshort{qict} clients by providing them with real-time data and visibility into their organizations,
      enabling them to make informed decisions and take proactive measures to protect their own digital assets and
      mitigate security risks.

      \section{Problem Statement}
      The company currently manages numerous third-party \acrshort{api}s for the above-mentioned purposes.
      Currently,
      Thus, the company has tasked the author with the development of the SentinelOne security threat platform integration
      for continuous cybersecurity monitoring within the \acrshort{qaas} app as the main topic of his graduation work placement
      project.

      \section{Project Objectives}
      In the end of this project which consist of 90-99 working days, the following objectives should be achieved:
      \begin{enumerate}
            \item Effectively integrates and leverages the SentinelOne \acrshort{edr} platform for continuous
                  cybersecurity monitoring within the \acrshort{qaas} app.
            \item The \acrshort{qaas} app should have a way to visualize the data retrieved from the SentinelOne
                  \acrshort{api} in a user-friendly manner in order for the client users helpdesk support, financial
                  department, cybersecurity department, software development department, and other employees within
                  \acrshort{qict}  departments to see the data easier.
            \item Combine the SentinelOne data with N-Central \acrshort{api}
            \item Utilize Vigilance package of SentinelOne
            \item Ensure proper unit testing, code refactoring, commenting, and adherence to the overall code
                  conventional guidelines and best practices in both the test and live environments of the
                  \acrshort{qaas} app.
      \end{enumerate}
      \section{Reading Guide}
      This report is structured as follows:
      \begin{itemize}
            \item \textbf{Summary}: provides a brief and concise overview of the entire report, including the
                  research questions, key findings, and conclusion. Its purpose is to provide readers with a
                  quick and comprehensive understanding of the report.
            \item \textbf{Introduction}: provides an overview of the project background, the research topic of the
                  company, and the project objectives. It introduces the context of the research and outlines the
                  structure of the report.
            \item \textbf{Research}: presents the research results, including the research methodology, the findings,
                  and the analysis of the research questions. Firstly, it describes the methodologies employed
                  during the research, and then it provides a detailed account of the research process and the
                  outcomes of the research.
            \item \textbf{Realization}: provides a detailed description of the software end-product developed during
                  this work placement project. It outlines insights into design, development, and implementation
                  phases. It also highlights key features, functionalities, and technology specifications used in
                  the project.
            \item \textbf{Conclusions and Recommendations}: the Conclusion summarizes the key findings and results
                  achieved during the research and realization phases. The Recommendation section outlines the
                  proposed next steps and future research areas to further enhance the project and address any
                  outstanding issues. It discusses potential areas for further exploration or refinement.
            \item \textbf{References}: lists all the sources cited in the report following the appropriate to
                  \acrshort{apa} 7th edition citation style.
            \item \textbf{Appendices}: includes any additional supplementary information, data, or materials that
                  are relevant to the report but not included in the main body of the report.
      \end{itemize}
\end{multicols}