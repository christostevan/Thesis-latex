\chapter{Introduction}
\begin{multicols}{2}
    [\section{Project Background}]
    In today's rapidly evolving digital landscape, cybersecurity remains a paramount concern for organizations
    across all industries. With the proliferation of sophisticated cyber threats and the increasing complexity of
    \acrshort{it} infrastructures, business are constantly seeking new and innovative ways to protect their digital assets and
    fortify their defences and safeguard sensitive data. In this pursuit, cybersecurity consultant firms have
    emerged as a critical ally for organizations, providing expert guidance and support in the development and
    implementation of robust cybersecurity strategies, playing a pivotal role in offering expertise and guidance to
    help organizations navigate the intricate realm of cybersecurity.

    One of the key strategies employed by cybersecurity consultants is the integration of third-party security
    \acrshort{api}s into their arsenal of tools and technologies. These \acrshort{api}s provide invaluable
    functionalities, ranging from vulnerability assessment and security scans to device health monitoring and threat
    intelligence analysis by \acrshort{ai}. By leveraging these \acrshort{api}s, cybersecurity consultants can enhance
    their capabilities and provide a more comprehensive and effective security solution to their clients, streamline
    their operations, provide clients with robust, proactive security measurers, and improve their overall service delivery.

    The company in question, \acrshort{qict}, currently does not have a standardized \acrshort{api} management implementation
    within its internal application, the \acrshort{qaas} app. This has led to a several problems, namely:
    \begin{itemize}
        \item Inefficient and fragmented approach to \acrshort{api} management.
        \item Lack of user-friendliness, and slow and unclear navigation.
        \item Inconsistent integration of \acrshort{api}s into the application.
        \item Poses a significant challenge in error handling and debugging, as disparate error reporting mechanism
              across the APIs hinder efficient troubleshooting and resolution processes.
        \item Difficulty to maintain and update \acrshort{api}s.
        \item Lack of clear and concise documentation.
        \item Lack of a centralized \acrshort{api} management system.
        \item Inadequate security measures, as the company risks Inconsistent data retrieval and analysis across its \acrshort{api}s,
              potentially leading to incomplete insights into client \acrshort{it} systems and infrastructure..
    \end{itemize}
\end{multicols}
\section{Project Objectives}
In the end of this project, the following objectives should be achieved:
\begin{enumerate}
    \item Develop a
\end{enumerate}
