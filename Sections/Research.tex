\chapter{Research Results}
\begin{multicols}{2}
      \section{Research Topic}
      In a research, it is paramount to have the formulation of a clear research topic, research main question,
      and research sub-questions. The main question serves as the focal point around which the research revolves,
      encapsulating the primary objective or purpose of the study.
      The following main research question will be used throughout the research:
      \begin{center}
            \textit{"How can Q-ICT effectively enhance API monitoring within its internal application while
                  integrating and leveraging SentinelOne security threat platform for continuous cybersecurity
                  monitoring while still ensuring adherence to the highest security standards?"}
      \end{center}
      The research sub-questions are then used to function as a pathway that dissect the main
      question into smaller, more manageable components, which can then be addressed individually. This approach
      allows for a more comprehensive and in-depth analysis of the research topic, ensuring that all relevant
      aspects are covered and that the research is conducted in a systematic and organized manner.
      This research main question is therefore expanded in the following research sub-questions:
      \begin{itemize}
            \item What is the current situation of the \acrshort{qaas} app of \acrshort{qict}?
            \item What functionalities should be prioritized in the development of monitoring and managing
                  third-party \acrshort{api}s within an internal application while ensuring real-time monitoring,
                  error detection, and insight generation regarding \acrshort{api} connections?
            \item How can SentinelOne can be integrated into the \acrshort{qaas} app environment, especially
                  aligning with the \acrshort{api} monitoring functionality, while still utilizing their key
                  features and capabilities in context of cyber threat detection and remote IT infrastructure
                  management?
            \item What are the most suitable visualization techniques for displaying the data processed and
                  received by the \acrshort{qaas} app in \acrshort{xml} and \acrshort{json} formats to ensure
                  clear and insightful representation of threats detected by SentinelOne and other relevant
                  \acrshort{api} connections?
      \end{itemize}
      \section{Research Methodology}
      In this research, different research methods have been used to answer the research questions. This research
      will be based on the six \acrshort{ict} research methods defined by HBO-I (\cite{ictresearchmethods}). A
      research method for each sub-question is then defined along with how the results are considered valid and
      reliable:
      \subsection{Method of Data Collection}
      \begin{itemize}[label=-]
            \item Sub-question \#1:
            \item Sub-question \#2:
            \item Sub-question \#3:
            \item Sub-question \#4:
      \end{itemize}
      \subsection{Selected Measuring Instruments}
      \begin{itemize}[label=-]
            \item Sub-question \#1:
            \item Sub-question \#2:
            \item Sub-question \#3:
            \item Sub-question \#4: the selected measuring instruments for this sub-question will be through
                  observation and reading the studies of.
      \end{itemize}
      \subsection{Method of Data Analysis}
      \begin{itemize}[label=-]
            \item Sub-question \#1:
            \item Sub-question \#2:
            \item Sub-question \#3:
            \item Sub-question \#4:
      \end{itemize}
      \subsection{Reliability, Validity, and General Applicability}
      \begin{itemize}[label=-]
            \item Sub-question \#1:
            \item Sub-question \#2:
            \item Sub-question \#3:
            \item Sub-question \#4:
      \end{itemize}
      \section{Research Question \#1}
      The \acrshort{qaas} app is a web application that is used by \acrshort{qict}. It is in Dart with Flutter as
      the front-end framework.

      \subsection{QaaS App Infrastructure}
      \textbf{Flutter}

      Flutter is an open-source framework made by Google in 2017. It used as an \acrshort{ui} toolkit for building
      natively compiled applications for mobile, web, and desktop (Windows, mac\acrshort{os}, Linux) from a single
      codebase (\cite{flutter}).

      \textbf{Firebase}

      Firebase is a comprehensive platform for developing and managing web and mobile applications, created by
      Google and is party of \acrshort{gcp}. It was originally an independent company founded by Firebase, Inc.
      in 2011. It was then acquired by Google in 2014. Since then, it has become an integral part of Google's
      broader ecosystem of cloud services.

      Firebase is a \acrshort{baas} that provides developers with a variety of tools and services to help with both
      back-end infrastructure and front-end capabilities without worrying about managing servers or infrastructures.
      \begin{itemize}
            \item Firestore Database: Firestore is a NoSQL database that is part of the Firebase platform. It is a
                  flexible, scalable database for mobile, web, and server development. It keeps data in sync across
                  client apps through real-time listeners and offers offline support for mobile and web, so the
                  developers can build responsive apps that work regardless of network latency or Internet
                  connectivity.
            \item Authentication
            \item Functions
            \item Hosting
            \item Real-time database:
      \end{itemize}

      \textbf{Algolia}

      Algolia is used for search functionality. It is a search-as-a-service platform that enables developers to
      integrate and build fast, relevant search functionality into their applications and websites (\cite{algolia}).
      It provides a range of features and capabilities for building and managing search functionality, including
      full-text search, typo tolerance, and relevance tuning, as well as analytics and monitoring tools to help
      developers understand how users are interacting with their search functionality in real-time.

      The reason as to why \acrshort{qict} uses Algolia is that the nature of Firebase search engine is quite often
      proven to be inaccurate and slow.

      \textbf{NoSQL Database}

      \subsection{Q-ICT Internal APIs}
      Those \acrshort{api}s are the following:
      \begin{itemize}
            \item Resello: is used for \acrshort{qict} Microsoft subscriptions owned by Pax8. It is a cloud
                  marketplace that simplifies the way \acrshort{sme}s buy, sell, and manage cloud solutions through
                  automation. It provides a single platform to manage the entire cloud customer lifecycle, from
                  quote to cash to support, thus simplifying the process of buying, selling and managing cloud
                  solutions.
            \item SnelStart: is used for \acrshort{qict} automation of financial and accounting system software,
                  such as managing invoices, etc., for \acrshort{sme}s. It offers a range of products and services
                  to help businesses manage their finances, including accounting software, invoicing software, and
                  financial management tools.
            \item SentinelOne: is a cybersecurity platform that provides endpoint protection, detection, and
                  response capabilities to help organizations defend against advanced cyber threats. It leverages
                  \acrlong{ai} and machine learning to analyze and respond to security threats in real-time,
                  providing organizations with comprehensive protection against malware, ransomware, and other
                  cyber threats. It also provides visibility into clients' \acrshort{it} systems and infrastructure,
                  enabling organizations to gain insights into potential security risks and vulnerabilities and take
                  proactive measures to address them.
            \item Bodyguard.io: is used for security tab. It is a product from a Dutch company that filters and
                  scrutinizes downloads from web browsers to detect and prevent malicious files with real-time
                  download scanning capabilities.
            \item N-Central: is a product from N-Able and is used for monitoring clients' devices and ensuring the
                  overall security of their systems, \acrshort{it} infrastructure, and digital assets. It is a
                  \acrshort{rrm} platform designed to help \acrshort{msp} and \acrshort{it} professionals to
                  remotely monitor and manage their clients' devices and networks. It provides a comprehensive
                  set of tools and features for monitoring, managing, and securing clients' devices and networks,
                  including remote monitoring and management, patch management, antivirus, backup and disaster
                  recovery, and network topology mapping.
            \item PerfectView: is a \gls{CRM} application from a Dutch company designed to help manage, track, and
                  store information related to \acrshort{qict}'s current and potential customers.
            \item Computicate:
      \end{itemize}
      \section{Research Question \#2}
      \section{Research Question \#3}
      \section{Research Question \#4}
\end{multicols}