\chapter*{Summary}

% The field of cybersecurity is a rapidly evolving field, with new threats and vulnerabilities being discovered every day.
% The traditional approach to cybersecurity is to have a firewall, antivirus, and a backup solution. However, with the rise of
% more sophisticated threats, more cyber actors are looking for the new vulnerabilities in a network. This means that malware
% too, is evolving. There are now an increased in sophistication, using advance techniques to evade detection, more targeted
% attacks, aiming for specific industries or organizations to maximize damage or achieve specific goals. Thus, cybersecurity
% is becoming more and more relevant to business organizations, primarily because a successful cyberattack can cause severe
% damage to an organization, whether it is on their reputation or financial loss. Small businesses are not immune to this, and
% they are often the target of cyberattacks due to their lack of resources to invest in cybersecurity, especially when they
% do not have their employees trained in cybersecurity.

\acrshort{qict} is a small cybersecurity consultant firm based in Emmen, Drenthe, the Netherlands. They act as a trusted IT
provider that will handle all clients' IT infrastructure. The services they offered are network, storage and server management,
reselling Microsoft 365 subscriptions as an official partner of Microsoft, data backup and recovery, financial data analytics to
help clients gain valuable insights into their business, cloud migration and services, technical remote and on-site support by
their helpdesk services, customized software development and system integration, cybersecurity consulting and strategies, and
more. They have over 500 clients, all spread throughout the Netherlands, and their main target audience is small to medium-sized
businesses, ranging from the employee size of 1-100, that do not have an IT department, therefore lacking the technical knowledge. Q-ICT
is therefore given the responsibility to manage and securing clients IT infrastructure, and clients will ask for their supports in
case of technicality issues arise.

Q-ICT have their own internal system made by themselves, called the \acrshort{qaas} App, which is a web application integrated with
various cybersecurity,  management and financial tools that \acrshort{qict} uses to manage their business processes regarding their clients,
subscriptions, products, and services. It has 5 \acrshort{api}s connected to it. Together, these \acrshort{api}s provide \acrshort{qict} with
the tools to be used by both \acrshort{qict} employees and the clients. A client of Q-ICT can access the \acrshort{qaas} app and see the status
of their network, security of their endpoints, their financial status, their subscriptions within Microsoft 365, etc. The QaaS App therefore
acts as a customer insight management tool for the clients of Q-ICT. For the employees of Q-ICT it is a centralized operation platform designed
to streamline and manage all of Q-ICT's day-to-day business activities. Last year, \acrshort{qict} has purchased the rights to a SentinelOne
subscription with the help of one of their partner companies, Pax8. Since then, they have been utilizing this product to provide better endpoint
security to their clients, migrating from Bitdefender, a traditional antivirus solution, into more modern solutions such as EDR system provided
by SentinelOne.

SentinelOne is an American cybersecurity company that has a product of the same name. It is an Endpoint Detection and Response platform, which
purpose is to detect and respond to threats of an endpoint in real-time, replacing the traditional antivirus. Its Agents, which serves as
a next generation antivirus, utilize \acrshort{ai} and \acrshort{ml} to detect and respond to threats, and it is deployed on the endpoint itself.
Traditional antivirus uses methods that often struggles against new, unknown (zero-day) threats and malware that changes frequently. SentinelOne,
on the other hand, and other similar next-generation antivirus solutions use behavioural analysis to identify malware based on its actions rather than its code.
This allows them to detect and respond to new, unknown threats that traditional \acrshort{av} solutions would miss.
The purpose of an \acrshort{edr} platform is to have a centralized location to monitor and respond to threats on all endpoints in a network,
in a company. With this new cybersecurity tools in their system, Q-ICT hope to gives one of the best world-class cybersecurity solutions from Gartner
magic quadrant to small and medium business enterprises in the Netherlands, therefore contributing to the improvement of cybersecurity standards
in the Netherlands.

Q-ICT then also would like to take one step further and provide transparency to their client to maintain the trust in their relationship with
their clients and reinforcing their confidence in Q-ICT's services by displaying client's SentinelOne data exclusively to them in the QaaS App.
This integration of SentinelOne to the QaaS App will prevent Q-ICT from creating accounts to each user of their 500 client companies, and
increasing the sell value of the QaaS App. Q-ICT also would like to have more control over what data to display to the clients, as almost all
of them are non-technical users, therefore providing them with a dashboard that is more user-friendly with visualization on the data to
simplify them to the common users. Q-ICT also would like to have more control on adding additional capabilities on the QaaS App based on
what SentinelOne system can do, such as adding files into blocklist or whitelist, remediate and rollback an endpoint operating system,
network isolation, and more.

The integration of the SentinelOne \acrshort{api} into the \acrshort{qaas} app is possible through SentinelOne \acrshort{rest} \acrshort{api},
with the use of \acrshort{qaas} app existing framework and services, Flutter and Firebase. Once a SentinelOne Agent is properly installed on an
endpoint, a SentinelOne token from the parent environment with all the read permission needs to be created and stored in Google Secret Manager,
where it is there in the cloud, secured and encrypted. An API rate limiting needs to be created to not flood the API server with requests each
time a user refreshes the page. After that, client's SentinelOne data needs to be connected with their already existing  information in the
QaaS environment, proper limitations need to be set in the user's permission to prevent them from accessing data that do not belong to them,
especially when doing filtering or searching. The new module needs to be deployed in the live environment and tested with client's real data
and elevated privileges in the SentinelOne environment.

This project would be the first iteration to the integration of SentinelOne module to the QaaS App. Further user pilot testing by the real
clients and additional requirements from the inside market are recommended to further improve the module and add additional features.

% Therefore, required Firebase Cloud Functions are created to interact with the SentinelOne
% \acrshort{api} to retrieve the data from SentinelOne, and store it in Firebase Firestore. The Flutter client then will interact with the
% Firestore to retrieve the data from SentinelOne and display it to the user.

% Based on comparing to other existing \acrshort{edr} platforms, a visualization is necessary, to provide a better understanding of the data.
% The visualization should be as simple as possible, so as not to overwhelm the user with too much information. The users should be able to
% customize the visualization widgets to their liking, choosing what kind of data they want to see, and how they want to see it. The visualization
% regarding SentinelOne should be able to show the status of the endpoints, the threats detected, and the actions taken by the SentinelOne Agent.
% The type of data that can be visualized are limited, normally it requires an index and a value (\acrshort{eg} number or date time) to be visualized.

