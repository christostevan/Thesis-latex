\chapter{Research Design}
\begin{multicols}{2}
      \section{Research Topic}
      In research, it is paramount to have the formulation of a clear research topic, research main question,
      and research sub-questions. The main question serves as the focal point around which the research revolves,
      encapsulating the primary objective or purpose of the study.
      The following main research question will be used throughout the research:
      \addcontentsline{toc}{subsection}{How can Quality ICT B.V. effectively integrate and leverage SentinelOne EDR platform
            for continuous cybersecurity monitoring?}
      \begin{center}
            \textit{"How can Quality ICT B.V. effectively integrate and leverage SentinelOne EDR platform
                  for continuous cybersecurity monitoring?"}
      \end{center}
      \section{Research Sub-Questions}
      The research sub-questions are then used to function as a pathway that dissects the main
      question into smaller, more manageable components, which can then be addressed individually. This approach
      allows for a more comprehensive and in-depth analysis of the research topic, ensuring that all relevant
      aspects are covered and that the research is conducted in a systematic and organized manner.
      This research main question is therefore expanded in the following research sub-questions:
      \addcontentsline{toc}{subsection}{Sub-question \#1: What is the current situation of the QaaS app of
            Quality-ICT B.V.}
      \addcontentsline{toc}{subsection}{Sub-question \#2: What is SentinelOne EDR platform?}
      \addcontentsline{toc}{subsection}{Sub-question \#3: How can SentinelOne be integrated into the QaaS app
            environment?}
      \addcontentsline{toc}{subsection}{Sub-question \#4: What are the most suitable visualization techniques for
            displaying the data received by SentinelOne API in Flutter?}
      \begin{enumerate}
            \item What is the current situation of the \acrshort{qaas} app of \acrlong{qict} \acrshort{bv}?
            \item What is SentinelOne \acrshort{edr} platform, and how does its integration enhance the cybersecurity
                  monitoring capabilities of the \acrshort{qaas} app?
            \item How can SentinelOne be integrated into the \acrshort{qaas} app environment, while still
                  utilizing their key features and capabilities in context of cyber-threat detection and
                  remote \acrshort{it} infrastructure management?
            \item What are the most suitable visualization techniques for displaying the data processed and
                  received by  SentinelOne \acrshort{api} in Flutter compare to other Security Threat Platforms to
                  ensure clear and insightful representation of threats?
      \end{enumerate}
      \section{Research Methodology}
      In this research, different research methods have been used to answer the research questions. This research
      will be based on the six \acrshort{ict} research methods defined by HBO-I (\cite{ictresearchmethods}). A
      research method for each sub-question is then defined along with how the results are considered valid and
      reliable:
      \subsection{Method of Data Collection}
      \begin{itemize}[label=-]
            \item Sub-question \#1: desk research of Literature Study will be conducted, with the goal of creating
                  infrastructure information that displays the structure of the \acrshort{qaas} app and all its
                  dependencies. Furthermore, Interview with key stakeholders involved in the  development, maintenance,
                  and usage of the \acrshort{qaas} app will be conducted to gain insights into the current situation
                  of the app.
            \item Sub-question \#2: Literature Study on various articles on the Internet, interviews, expert reviews,
                  and requirement elicitation techniques such as use case analysis and user stories. Analysis on the
                  current \acrshort{qaas} app and its capabilities.
            \item Sub-question \#3: technical assessments and will be conducted, under the supervision of the Company
                  Supervisor, to evaluate the technical feasibility, compactibility, and alignment of SentinelOne's
                  features with the \acrshort{qaas} app environment.
            \item Sub-question \#4: research into existing visualization techniques for \gls{JSON} data coming from
                  SentinelOne \acrshort{api} through Literature Study. Analyze existing data visualization tools and
                  platforms that are available in Flutter and Firebase. Gather requirements from project stakeholders
                  regarding data visualization preferences and usability, and do data analysis and usability testing.
                  % \item Sub-question \#4: technical evaluations of SentinelOne's capabilities and \acrshort{api}s will
                  %       be conducted, the \acrshort{api} documentation and integration guideline will be read and review
                  %       with Literature Study method and case studies of similar integrations. Requirements from the
                  %       cybersecurity experts from the \acrshort{qict} department responsible for SentinelOne's technical
                  %       support will be gathered and analyzed. Furthermore, Prototyping with proof-of-concept prototypes
                  %       on a test environment will be conducted to test different integration scenarios, assess feasibility,
                  %       identify potential challenges, refine the approach, and evaluate the performance of the integration.
                  \subsection{Selected Measuring Instruments}
      \end{itemize}
      \begin{itemize}[label=-]
            \item Sub-question \#1: structured interview guide, document report checklist analysis and review,
                  observation, analysis tools for codebase and logs, and quite possibly supplemented by surveys
                  or questionnaires.
            \item Sub-question \#2: document analysis tools for literature review. Structured questionnaires for
                  requirement interviews regarding functionality rating scale and compare the response against
                  industry standards and best practices. Observation of existing \acrshort{api} monitoring tools.
                  Prioritize functionalities based on importance, feasibility, and impact on the \acrshort{qaas} app.
            \item Sub-question \#3: technical assessments and requirement workshops will be conducted. Furthermore,
                  \acrshort{api} documentation  review, document analysis tools, security impact risk assessment,
                  and feasibility checklist assessment with the Company Supervisor will also be overseen.
            \item Sub-question \#4: the selected measuring instruments for this sub-question will be through
                  observation of existing data visualization tools, literature review through reading the studies of
                  the best visualization suitability matrix techniques, structured questionnaires to end-user
                  interviews, and usability testing heuristics.
      \end{itemize}
      \subsection{Method of Data Analysis}
      \begin{itemize}[label=-]
            \item Sub-question \#1: a qualitative thematic \acrshort{swot} analysis of interview transcripts
                  and documentation for operational insights to identify strengths, weaknesses, and areas for
                  improvement in the current situation of the \acrshort{qaas} app.
            \item Sub-question \#2: comparative analyze survey/interview responses using \acrshort{mcda} and
                  compare against industry standards and best practices. Prioritize functionalities based on
                  importance, feasibility, and impact on the \acrshort{qaas} app.
            \item Sub-question \#3: evaluate the technical feasibility, compactibility, and alignment of
                  SentinelOne's features with the \acrshort{qaas} app environment. Analyze potential integration
                  challenges and mitigation strategies and assess the performance of the integration through
                  prototyping and testing. Technical analysis for the \acrshort{api} documentation and
                  thematic analysis for interview data.
            \item Sub-question \#4: technical tool analysis by reviewing and evaluating the suitability of different
                  visualization techniques for representing the data processed and received by the \acrshort{qaas}
                  app in \acrshort{xml} and \acrshort{json} formats from the \acrshort{api} considering factors such
                  as clarity, interpretability, and user engagement. Do a user-centered design and cognitive load
                  analysis by analyzing feedback from company stakeholders, supervisor, and end-users.
      \end{itemize}
      \subsection{Reliability, Validity, and General Applicability}
      \begin{itemize}[label=-]
            \item Sub-question \#1: the reliability of the data can be ensured by triangulation of data from
                  multiple sources and conducting interviews with stakeholders from different departments with
                  structure questionnaires to ensure that the data is consistent and accurate. The validity of
                  the data will be ensured by  cross-referencing with the existing literature or industry best
                  practices or other sources and  through information obtained from  interviews with the
                  \acrshort{qaas} app developers to ensure  that the data is accurate and reliable. The general
                  applicability of the data will be ensured by  ensuring that the information obtained is relevant
                  and applicable to the research question and  that it can be used to draw meaningful conclusions
                  and make informed decisions, furthermore by comparing findings with industry standards and best
                  practices or similar case studies or projects.
            \item Sub-question \#2: ensure reliability through sampling techniques and representative stakeholder
                  involvement, with comprehensive literature review and multiple sources of information. Validate
                  priorities against real-world scenarios or case studies involving diverse expert panel, like
                  the Company Supervisor. General applicability can be assessed by comparing prioritization with
                  similar projects or frameworks, and considering scalability and adaptability of the integration
                  with representative user samples.
            \item Sub-question \#3: the validity of this sub-question will be through pilot integration unit testing
                  or proof of concept documents and ensuring alignment with cybersecurity standards and best
                  practices. The reliability will be to consider future needs such as adaptability and scalability of
                  the integration, and focus on \acrshort{qict} user context and needs. General applicability can be
                  assessed by comparing integration strategies with industry standards or expert opinions such as from
                  the Company Supervisor.
            \item Sub-question \#4: reliability can be defined by ensuring future adaptability with comprehensive
                  literature review and multiple sources of information. Validity can be achieved through validating
                  visualization choices through data-driven approach in usability testing or  prototyping, ensuring
                  alignment with best practices in data visualization and involving the expert, like the Company
                  Supervisor on the field. General applicability can be assessed by accessibility considerations by
                  comparing proposed visualization techniques with similar applications or domains.
      \end{itemize}
      \subsection{Research Limitations}
      The project and research in general will be limited on the \acrshort{api} request methods, in which the author
      is allowed to do only GET requests. This is due to the fact that the author is not allowed to do any PATCH,
      POST, PUT, DELETE, or any other \acrshort{http} request methods that could potentially change the state of the
      \acrshort{qaas} app or the \acrshort{api} that is being requested. This limitation is because the author is not a
      full-time employee of \acrshort{qict} and is not allowed to make any changes to the \acrshort{qaas} app or the
      \acrshort{api} that is being requested. Therefore, the author is limited to do research in the best practices
      of SentinelOne integration for the GET request method only.

      The author is also limited in showing the SentinelOne dashboard data, as it contains clients' sensitive
      information, and \acrshort{qict} has over 400 clients. Therefore, if any part of the SentinelOne dashboard is
      shown, it will be with blurred sensitive information.

      Moreover, the author is also limited to the non-disclosure agreement signed within the initialization period of
      the graduation work placement. This means that any confidential information that the company deems as confidential
      will not be disclosed in this research. This includes any information that is not publicly available, such as any
      financial data or security data pertaining to the internal system or the \acrshort{qaas} app internal code.
\end{multicols}
